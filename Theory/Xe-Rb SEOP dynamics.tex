\documentclass{article}

% Language setting
% Replace `english' with e.g. `spanish' to change the document language
\usepackage[english]{babel}

% Set page size and margins
% Replace `letterpaper' with`a4paper' for UK/EU standard size
\usepackage[letterpaper,top=2cm,bottom=2cm,left=3cm,right=3cm,marginparwidth=1.75cm]{geometry}

% Useful packages
\usepackage{amsmath}
\usepackage{graphicx}
\usepackage[colorlinks=true, allcolors=blue]{hyperref}

%\title{Your Paper}
%\author{You}

\begin{document}
\title{Xe129 - Rb87 Spin Exchange Optical Pumping dynamics}
\maketitle


% \begin{abstract}
% Your abstract.
% \end{abstract}

\section{Bloch equations}
The Block equation for the Xe spin are

\begin{align}
    \frac{d \mathbf{K}}{d t} &= -\left[\gamma_{Xe}\left( \mathbf{B}_0 + \mathbf{B}_d \cos{(\omega_{rf} t)}\right) + b_{KS} \mathbf{S}\right] \times\mathbf{K}  +\Gamma_{se} \left( \mathbf{S} - \mathbf{K}\right) - \Gamma_2 \hat{x}\left(\hat{x}\cdot \mathbf{K}\right)  - \Gamma_2 \hat{y}\left(\hat{y}\cdot \mathbf{K}\right) -  \Gamma_1 \hat{z}\left(\hat{z}\cdot \mathbf{K}\right)
\end{align}

Let us assume that $b_{KS} \mathbf{S}$ is part of the DC field $\mathbf{B}_0=B_0 \hat{z}$, that $\Gamma_{se} \left( \mathbf{S} - \mathbf{K}\right)=\left(\begin{array}{c}0\\0\\R_{se}\end{array}\right)$ and that the drive is $$\mathbf{B}_d \cos{(\omega_{rf} t)} =B_d \cos{(\omega_{rf} t)}\hat{x}$$ .

Thus, we end up with the next Bloch equation
\begin{align}
    \frac{d \mathbf{K}}{d t} &= -\left[\gamma_{Xe}\left( B_0\hat{z} + B_d \cos{(\omega_{rf} t)}\hat{x}\right) \right] \times\mathbf{K}  +R_{se}\hat{z} - \Gamma_2 \hat{x}\left(\hat{x}\cdot \mathbf{K}\right)  - \Gamma_2 \hat{y}\left(\hat{y}\cdot \mathbf{K}\right) -  \Gamma_1 \hat{z}\left(\hat{z}\cdot \mathbf{K}\right)\\
    &= -\gamma_{Xe} \begin{vmatrix}
        \hat{x}                     & \hat{y}       & \hat{z} \\
         B_d\cos{(\omega_{rf} t)}   & 0             & B_0  \\
         K_x                        & K_y           & K_z
    \end{vmatrix} + 
    \left(\begin{matrix}
         0  \\
         0  \\
         R_{se} 
    \end{matrix}\right) - 
    \left(\begin{matrix}
    \Gamma_2  &  0  &0\\
    0  &  \Gamma_2  &  0\\
    0  &  0  &  \Gamma_1 
    \end{matrix}\right)\cdot
    \left(\begin{matrix}
    K_x\\
    K_y\\
    K_z
    \end{matrix}\right)\\
    &= - \gamma_{Xe} 
    \left(\begin{matrix}
    -B_0 K_y\\
    B_0 K_x - B_d \cos{(\omega_{rf} t)} K_z\\
    B_d \cos{(\omega_{rf} t)} K_y
    \end{matrix}\right) + 
    \left(\begin{matrix}
         0  \\
         0  \\
         R_{se} 
    \end{matrix}\right) - 
    \left(\begin{matrix}
    \Gamma_2  &  0  &0\\
    0  &  \Gamma_2  &  0\\
    0  &  0  &  \Gamma_1 
    \end{matrix}\right)\cdot
    \left(\begin{matrix}
    K_x\\
    K_y\\
    K_z
    \end{matrix}\right)\\
    &=  
    \left(\begin{matrix}
    -\Gamma_2               &  \gamma_{Xe}B_0                        &  0                                    \\
    -\gamma_{Xe}B_0         &  -\Gamma_2                             &  \gamma_{Xe}B_d \cos{(\omega_{rf} t)} \\
    0                       &  -\gamma_{Xe}B_d \cos{(\omega_{rf} t)} &  -\Gamma_1 
    \end{matrix}\right)\cdot
    \left(\begin{matrix}
    K_x\\
    K_y\\
    K_z
    \end{matrix}\right) + 
    \left(\begin{matrix}
         0  \\
         0  \\
         R_{se} 
    \end{matrix}\right)\\
    &=  
    \left(\begin{matrix}
    -\Gamma_2               &  \omega_0                        &  0                              \\
    -\omega_0               &  -\Gamma_2                       &  \Omega_d \cos{(\omega_{rf} t)} \\
    0                       &  -\Omega_d \cos{(\omega_{rf} t)} &  -\Gamma_1 
    \end{matrix}\right)\cdot
    \left(\begin{matrix}
    K_x\\
    K_y\\
    K_z
    \end{matrix}\right) + 
    \left(\begin{matrix}
         0  \\
         0  \\
         R_{se} 
    \end{matrix}\right)
\end{align}

Finally, the Xe NMR dynamics can be described by the next Bloch equations

\begin{align}
    \left(\begin{matrix}
    \dot{K_x}\\
    \dot{K_y}\\
    \dot{K_z}
    \end{matrix}\right)
    &=  
    \left(\begin{matrix}
    -\Gamma_2               &  \omega_0                        &  0                              \\
    -\omega_0               &  -\Gamma_2                       &  \Omega_d \cos{(\omega_{rf} t)} \\
    0                       &  -\Omega_d \cos{(\omega_{rf} t)} &  -\Gamma_1 
    \end{matrix}\right)\cdot
    \left(\begin{matrix}
    K_x\\
    K_y\\
    K_z
    \end{matrix}\right) + 
    \left(\begin{matrix}
         0  \\
         0  \\
         R_{se} 
    \end{matrix}\right)
\end{align}


Now, let us move to the rotating frame (frame of Xe) using the next transformation

\begin{align}
    \left(\begin{array}{c}
        \tilde{K}_{x}\\
        \tilde{K}_{y}\\
        \tilde{K}_{z}
    \end{array}\right)	&=\left(\begin{array}{ccc}
        \cos(\omega_{0}t) & -\sin(\omega_{0}t) & 0\\
        \sin(\omega_{0}t) & \cos(\omega_{0}t) & 0\\
        0 & 0 & 1
    \end{array}\right)\left(\begin{array}{c}
        K_{x}\\
        K_{y}\\
        K_{z}
    \end{array}\right)\\
    \left(\begin{array}{c}
        K_{x}\\
        K_{y}\\
        K_{z}
    \end{array}\right)	&=\left(\begin{array}{ccc}
        \cos(\omega_{0}t) & \sin(\omega_{0}t) & 0\\
        -\sin(\omega_{0}t) & \cos(\omega_{0}t) & 0\\
        0 & 0 & 1
    \end{array}\right)\left(\begin{array}{c}
        \tilde{K}_{x}\\
        \tilde{K}_{y}\\
        \tilde{K}_{z}
    \end{array}\right)
\end{align}

Also, the derivative with respect to time of the transformation is
\begin{align}
    \left(\begin{array}{c}
        \dot{K_{x}}\\
        \dot{K_{y}}\\
        \dot{K_{z}}
    \end{array}\right)	&=\frac{d}{dt}\left[\left(\begin{array}{ccc}
        \cos(\omega_{0}t) & \sin(\omega_{0}t) & 0\\
        -\sin(\omega_{0}t) & \cos(\omega_{0}t) & 0\\
        0 & 0 & 1
    \end{array}\right)\left(\begin{array}{c}
        \tilde{K}_{x}\\
        \tilde{K}_{y}\\
        \tilde{K}_{z}
    \end{array}\right)\right]\\ 
    &=\frac{d}{dt}\left(\begin{array}{ccc}
        \cos(\omega_{0}t) & \sin(\omega_{0}t) & 0\\
        -\sin(\omega_{0}t) & \cos(\omega_{0}t) & 0\\
        0 & 0 & 1
    \end{array}\right)\left(\begin{array}{c}
        \tilde{K}_{x}\\
        \tilde{K}_{y}\\
        \tilde{K}_{z}
    \end{array}\right) + 
    \left(\begin{array}{ccc}
        \cos(\omega_{0}t) & \sin(\omega_{0}t) & 0\\
        -\sin(\omega_{0}t) & \cos(\omega_{0}t) & 0\\
        0 & 0 & 1
    \end{array}\right)
    \frac{d}{dt}\left(\begin{array}{c}
        \tilde{K}_{x}\\
        \tilde{K}_{y}\\
        \tilde{K}_{z}
    \end{array}\right)\\
    &=\omega_{0}\left(\begin{array}{ccc}
        -\sin(\omega_{0}t) & \cos(\omega_{0}t) & 0\\
        -\cos(\omega_{0}t) & -\sin(\omega_{0}t) & 0\\
        0 & 0 & 0
    \end{array}\right)
    \left(\begin{array}{c}
        \tilde{K}_{x}\\
        \tilde{K}_{y}\\
        \tilde{K}_{z}
    \end{array}\right) + 
    \left(\begin{array}{ccc}
        \cos(\omega_{0}t) & \sin(\omega_{0}t) & 0\\
        -\sin(\omega_{0}t) & \cos(\omega_{0}t) & 0\\
        0 & 0 & 1
    \end{array}\right)
    \frac{d}{dt}\left(\begin{array}{c}
        \tilde{K}_{x}\\
        \tilde{K}_{y}\\
        \tilde{K}_{z}
    \end{array}\right)
\end{align}



The Bloch equations transform as follows

\begin{align}
    \omega_{0}\left(\begin{array}{ccc}
        -\sin(\omega_{0}t) & \cos(\omega_{0}t) & 0\\
        -\cos(\omega_{0}t) & -\sin(\omega_{0}t) & 0\\
        0 & 0 & 0
    \end{array}\right)
    \left(\begin{array}{c}
        \tilde{K}_{x}\\
        \tilde{K}_{y}\\
        \tilde{K}_{z}
    \end{array}\right) + 
    \left(\begin{array}{ccc}
        \cos(\omega_{0}t) & \sin(\omega_{0}t) & 0\\
        -\sin(\omega_{0}t) & \cos(\omega_{0}t) & 0\\
        0 & 0 & 1
    \end{array}\right)
    \frac{d}{dt}\left(\begin{array}{c}
        \tilde{K}_{x}\\
        \tilde{K}_{y}\\
        \tilde{K}_{z}
    \end{array}\right)
    &=\\ 
    \left(\begin{matrix}
    -\Gamma_2               &  \omega_0                        &  0                              \\
    -\omega_0               &  -\Gamma_2                       &  \Omega_d \cos{(\omega_{rf} t)} \\
    0                       &  -\Omega_d \cos{(\omega_{rf} t)} &  -\Gamma_1 
    \end{matrix}\right)\cdot
    \left(\begin{array}{ccc}
        \cos(\omega_{0}t) & \sin(\omega_{0}t) & 0\\
        -\sin(\omega_{0}t) & \cos(\omega_{0}t) & 0\\
        0 & 0 & 1
    \end{array}\right)
    \left(\begin{array}{c}
        \tilde{K}_{x}\\
        \tilde{K}_{y}\\
        \tilde{K}_{z}
    \end{array}\right) &+ 
    \left(\begin{matrix}
         0  \\
         0  \\
         R_{se} 
    \end{matrix}\right)\\
    % new line
    \omega_{0}
    \left(\begin{array}{ccc}
        \cos(\omega_{0}t) & -\sin(\omega_{0}t) & 0\\
        \sin(\omega_{0}t) & \cos(\omega_{0}t) & 0\\
        0 & 0 & 1
    \end{array}\right)
    \left(\begin{array}{ccc}
        -\sin(\omega_{0}t) & \cos(\omega_{0}t) & 0\\
        -\cos(\omega_{0}t) & -\sin(\omega_{0}t) & 0\\
        0 & 0 & 0
    \end{array}\right)
    \left(\begin{array}{c}
        \tilde{K}_{x}\\
        \tilde{K}_{y}\\
        \tilde{K}_{z}
    \end{array}\right) + 
    \frac{d}{dt}\left(\begin{array}{c}
        \tilde{K}_{x}\\
        \tilde{K}_{y}\\
        \tilde{K}_{z}
    \end{array}\right)
    &=\\ 
    \left(\begin{array}{ccc}
        \cos(\omega_{0}t) & -\sin(\omega_{0}t) & 0\\
        \sin(\omega_{0}t) & \cos(\omega_{0}t) & 0\\
        0 & 0 & 1
    \end{array}\right)
    \left(\begin{matrix}
    -\Gamma_2               &  \omega_0                        &  0                              \\
    -\omega_0               &  -\Gamma_2                       &  \Omega_d \cos{(\omega_{rf} t)} \\
    0                       &  -\Omega_d \cos{(\omega_{rf} t)} &  -\Gamma_1 
    \end{matrix}\right)
    \left(\begin{array}{ccc}
        \cos(\omega_{0}t) & \sin(\omega_{0}t) & 0\\
        -\sin(\omega_{0}t) & \cos(\omega_{0}t) & 0\\
        0 & 0 & 1
    \end{array}\right)
    \left(\begin{array}{c}
        \tilde{K}_{x}\\
        \tilde{K}_{y}\\
        \tilde{K}_{z}
    \end{array}\right) &+ 
    \left(\begin{matrix}
         0  \\
         0  \\
         R_{se} 
    \end{matrix}\right)
\end{align}

Let us compute it separately term by term

\begin{align}
    \omega_{0}
    \left(\begin{array}{ccc}
        \cos(\omega_{0}t) & -\sin(\omega_{0}t) & 0\\
        \sin(\omega_{0}t) & \cos(\omega_{0}t) & 0\\
        0 & 0 & 1
    \end{array}\right)
    \left(\begin{array}{ccc}
        -\sin(\omega_{0}t) & \cos(\omega_{0}t) & 0\\
        -\cos(\omega_{0}t) & -\sin(\omega_{0}t) & 0\\
        0 & 0 & 0
    \end{array}\right)
    \left(\begin{array}{c}
        \tilde{K}_{x}\\
        \tilde{K}_{y}\\
        \tilde{K}_{z}
    \end{array}\right)&= 
    \omega_{0}
    \left(\begin{array}{ccc}
       0 & 1 & 0\\
       -1 & 0 & 0\\
        0 & 0 & 0
    \end{array}\right)
    \left(\begin{array}{c}
        \tilde{K}_{x}\\
        \tilde{K}_{y}\\
        \tilde{K}_{z}
    \end{array}\right)\\
    &=
    \omega_{0}
    \left(\begin{array}{c}
        \tilde{K}_{y}\\
        -\tilde{K}_{x}\\
        0
    \end{array}\right)
\end{align}

Also,

\begin{align}
    \left(\begin{array}{ccc}
        \cos(\omega_{0}t) & -\sin(\omega_{0}t) & 0\\
        \sin(\omega_{0}t) & \cos(\omega_{0}t) & 0\\
        0 & 0 & 1
    \end{array}\right)
    \left(\begin{matrix}
    -\Gamma_2               &  \omega_0                        &  0                              \\
    -\omega_0               &  -\Gamma_2                       &  \Omega_d \cos{(\omega_{rf} t)} \\
    0                       &  -\Omega_d \cos{(\omega_{rf} t)} &  -\Gamma_1 
    \end{matrix}\right)
    \left(\begin{array}{ccc}
        \cos(\omega_{0}t) & \sin(\omega_{0}t) & 0\\
        -\sin(\omega_{0}t) & \cos(\omega_{0}t) & 0\\
        0 & 0 & 1
    \end{array}\right) &=\\
    \left(\begin{array}{ccc}
        \cos(\omega_{0}t) & -\sin(\omega_{0}t) & 0\\
        \sin(\omega_{0}t) & \cos(\omega_{0}t) & 0\\
        0 & 0 & 1
    \end{array}\right)
    \left(\begin{matrix}
    -\Gamma_2 \cos(\omega_{0}t) -\omega_{0}\sin(\omega_{0}t) & -\Gamma_2 \sin(\omega_{0}t) +\omega_{0}\cos(\omega_{0}t)    &  0  \\
    -\omega_{0} \cos(\omega_{0}t) + \Gamma_2 \sin(\omega_{0}t)   &  -\omega_{0} \sin(\omega_{0}t) - \Gamma_2 \cos(\omega_{0}t)  &  \Omega_d \cos{(\omega_{rf} t)} \\ \Omega_d \cos{(\omega_{rf} t)}\sin(\omega_{0}t) &  -\Omega_d \cos{(\omega_{rf} t)}\cos(\omega_{0}t) &  -\Gamma_1 
    \end{matrix}\right) &=\\
    % last line
    \left(\begin{matrix}
    -\Gamma_2 & \omega_{0}    &  -\Omega_d \cos{(\omega_{rf} t)}\sin(\omega_{0}t) \\
    -\omega_{0} &  - \Gamma_2 &  \Omega_d \cos{(\omega_{rf} t)}\cos(\omega_{0}t) \\ \Omega_d \cos{(\omega_{rf} t)}\sin(\omega_{0}t) &  -\Omega_d \cos{(\omega_{rf} t)}\cos(\omega_{0}t) &  -\Gamma_1 
    \end{matrix}\right)
\end{align}

combining the terms together we get

\begin{align}
    \omega_{0}
    \left(\begin{array}{ccc}
       0 & 1 & 0\\
       -1 & 0 & 0\\
        0 & 0 & 0
    \end{array}\right)
    \left(\begin{array}{c}
        \tilde{K}_{x}\\
        \tilde{K}_{y}\\
        \tilde{K}_{z}
    \end{array}\right) + 
    \frac{d}{dt}\left(\begin{array}{c}
        \tilde{K}_{x}\\
        \tilde{K}_{y}\\
        \tilde{K}_{z}
    \end{array}\right)
    &=\\
   \left(\begin{matrix}
    -\Gamma_2 & \omega_{0}    &  -\Omega_d \cos{(\omega_{rf} t)}\sin(\omega_{0}t) \\
    -\omega_{0} &  - \Gamma_2 &  \Omega_d \cos{(\omega_{rf} t)}\cos(\omega_{0}t) \\ \Omega_d \cos{(\omega_{rf} t)}\sin(\omega_{0}t) &  -\Omega_d \cos{(\omega_{rf} t)}\cos(\omega_{0}t) &  -\Gamma_1 
    \end{matrix}\right)
    \left(\begin{array}{c}
        \tilde{K}_{x}\\
        \tilde{K}_{y}\\
        \tilde{K}_{z}
    \end{array}\right) &+ 
    \left(\begin{matrix}
         0  \\
         0  \\
         R_{se} 
    \end{matrix}\right)\\
    \frac{d}{dt}\left(\begin{array}{c}
        \tilde{K}_{x}\\
        \tilde{K}_{y}\\
        \tilde{K}_{z}
    \end{array}\right)
    &=\\
   \left(\begin{matrix}
    -\Gamma_2 & \omega_{0}    &  -\Omega_d \cos{(\omega_{rf} t)}\sin(\omega_{0}t) \\
    -\omega_{0} &  - \Gamma_2 &  \Omega_d \cos{(\omega_{rf} t)}\cos(\omega_{0}t) \\ \Omega_d \cos{(\omega_{rf} t)}\sin(\omega_{0}t) &  -\Omega_d \cos{(\omega_{rf} t)}\cos(\omega_{0}t) &  -\Gamma_1 
    \end{matrix}\right)
    \left(\begin{array}{c}
        \tilde{K}_{x}\\
        \tilde{K}_{y}\\
        \tilde{K}_{z}
    \end{array}\right) &+
    \left(\begin{array}{ccc}
       0 & -\omega_{0} & 0\\
       \omega_{0} & 0 & 0\\
        0 & 0 & 0
    \end{array}\right)
    \left(\begin{array}{c}
        \tilde{K}_{x}\\
        \tilde{K}_{y}\\
        \tilde{K}_{z}
    \end{array}\right)\\&+ 
    \left(\begin{matrix}
         0  \\
         0  \\
         R_{se} 
    \end{matrix}\right)
\end{align}

so finally

\begin{align}
    \frac{d}{dt}\left(\begin{array}{c}
        \tilde{K}_{x}\\
        \tilde{K}_{y}\\
        \tilde{K}_{z}
    \end{array}\right)
    &=
   \left(\begin{matrix}
    -\Gamma_2 & 0    &  -\Omega_d \cos{(\omega_{rf} t)}\sin(\omega_{0}t) \\
    0 &  - \Gamma_2 &  \Omega_d \cos{(\omega_{rf} t)}\cos(\omega_{0}t) \\ \Omega_d \cos{(\omega_{rf} t)}\sin(\omega_{0}t) &  -\Omega_d \cos{(\omega_{rf} t)}\cos(\omega_{0}t) &  -\Gamma_1 
    \end{matrix}\right)
    \left(\begin{array}{c}
        \tilde{K}_{x}\\
        \tilde{K}_{y}\\
        \tilde{K}_{z}
    \end{array}\right) + 
    \left(\begin{matrix}
         0  \\
         0  \\
         R_{se} 
    \end{matrix}\right)\\
\end{align}

Notice that in the case where we transfer to a different rotating frame of reference which rotates at some frequency $\omega_1$, then the Bloch equations would be a bit different after the transformation

\begin{align}
    \frac{d}{dt}\left(\begin{array}{c}
        \tilde{K}_{x}\\
        \tilde{K}_{y}\\
        \tilde{K}_{z}
    \end{array}\right)
    &=
   \left(\begin{matrix}
    -\Gamma_2 & \omega_0-\omega_1    &  -\Omega_d \cos{(\omega_{rf} t)}\sin(\omega_{1}t) \\
    -\left( \omega_0-\omega_1\right) &  - \Gamma_2 &  \Omega_d \cos{(\omega_{rf} t)}\cos(\omega_{1}t) \\ \Omega_d \cos{(\omega_{rf} t)}\sin(\omega_{1}t) &  -\Omega_d \cos{(\omega_{rf} t)}\cos(\omega_{1}t) &  -\Gamma_1 
    \end{matrix}\right)
    \left(\begin{array}{c}
        \tilde{K}_{x}\\
        \tilde{K}_{y}\\
        \tilde{K}_{z}
    \end{array}\right) + 
    \left(\begin{matrix}
         0  \\
         0  \\
         R_{se} 
    \end{matrix}\right)\\
\end{align}

Using trigonometric identities we can simplify the drive terms in the Bloch matrix

\begin{align}
    \Omega_d \cos{(\omega_{rf} t)}\sin(\omega_{1}t) &= \frac{\Omega_d}{2} \left(\sin(\left(\omega_{rf}+\omega_{1}\right)t)+\sin(\left(\omega_{rf}-\omega_{1}\right)t)\right)\\
    \Omega_d \cos{(\omega_{rf} t)}\cos(\omega_{1}t) &= \frac{\Omega_d}{2} \left(\cos(\left(\omega_{rf}-\omega_{1}\right)t)+\cos(\left(\omega_{rf}+\omega_{1}\right)t)\right)\\
\end{align}

assuming that $\omega_{rf} - \omega_{1} \ll 1$, then we can use the RWA which states that fast oscillations average out to zero

\begin{align}
    \Omega_d \cos{(\omega_{rf} t)}\sin(\omega_{1}t) &= \frac{\Omega_d}{2} \sin(2\omega_{rf}t)=0\\
    \Omega_d \cos{(\omega_{rf} t)}\cos(\omega_{1}t) &= \frac{\Omega_d}{2} \left(1+\cos(2\omega_{rf}t)\right)=\frac{\Omega_d}{2}\\
\end{align}

Then, the Bloch equations with the RWA are

\begin{align}
    \frac{d}{dt}\left(\begin{array}{c}
        \tilde{K}_{x}\\
        \tilde{K}_{y}\\
        \tilde{K}_{z}
    \end{array}\right)=
   \left(\begin{matrix}
    -\Gamma_2 & \omega_0-\omega_{rf}    &  0 \\
    -\left( \omega_0-\omega_{rf}\right) &  - \Gamma_2 &  \frac{\Omega_d}{2} \\ 0 &  -\frac{\Omega_d}{2}  &  -\Gamma_1 
    \end{matrix}\right)
    \left(\begin{array}{c}
        \tilde{K}_{x}\\
        \tilde{K}_{y}\\
        \tilde{K}_{z}
    \end{array}\right) + 
    \left(\begin{matrix}
         0  \\
         0  \\
         R_{se} 
    \end{matrix}\right)
\end{align}

Next, we add to our model The effective magnetic field the Xe particles fill which produced by the alkali atoms, $b_{KS}\mathbf{S}=b_{KS}S_z\hat{z}$ and also
some world rotation $\mathbf{\omega_r} = \omega_r\hat{z}$.

The modified equations are

\begin{align}
    \boxed{\frac{d}{dt}\left(\begin{array}{c}
        \tilde{K}_{x}\\
        \tilde{K}_{y}\\
        \tilde{K}_{z}
    \end{array}\right)=
   \left(\begin{matrix}
    -\Gamma_2 & M_{12}    &  0 \\
    M_{21} &  - \Gamma_2 &  \frac{\Omega_d}{2} \\ 0 &  -\frac{\Omega_d}{2}  &  -\Gamma_1 
    \end{matrix}\right)
    \left(\begin{array}{c}
        \tilde{K}_{x}\\
        \tilde{K}_{y}\\
        \tilde{K}_{z}
    \end{array}\right) + 
    \left(\begin{matrix}
         0  \\
         0  \\
         R_{se} 
    \end{matrix}\right)}\label{eq:Bloch_equations}
\end{align}

where $M_{21}=-M_{12}$ and
\begin{align}
    M_{12} &= \omega_{0}+\gamma_{Xe}b_{KS}S_z+\omega_r-\omega_{rf}\\
        &= \gamma_{Xe}B_0+\gamma_{Xe}b_{KS}S_z+\omega_r-\omega_{rf}\\
\end{align}

Next, let us find the steady state solution of the Bloch equations in the rotating frame with drive in $\hat{x}$ with the RWA

\begin{align}
    \left(\begin{matrix}
    \Gamma_2 & -M_{12}    &  0 \\
    M_{12} &   \Gamma_2 &  -\frac{\Omega_d}{2} \\ 0 &  \frac{\Omega_d}{2}  &  \Gamma_1 
    \end{matrix}\right)
    \left(\begin{array}{c}
        \tilde{K}_{x}\\
        \tilde{K}_{y}\\
        \tilde{K}_{z}
    \end{array}\right) = 
    \left(\begin{matrix}
         0  \\
         0  \\
         R_{se} 
    \end{matrix}\right)
\end{align}

\begin{align}
    \left(\begin{array}{c}
        \Gamma_2\tilde{K}_{x} - M_{12}\tilde{K}_{y}\\
        M_{12}\tilde{K}_{x}+\Gamma_2\tilde{K}_{y}-\frac{\Omega_d}{2}\tilde{K}_{z}\\
        \frac{\Omega_d}{2}\tilde{K}_{y}+\Gamma_1\tilde{K}_{z}
    \end{array}\right)&=
    \left(\begin{array}{c}
        0\\
        0\\
        R_{se} 
    \end{array}\right)
\end{align}

\begin{align}
    \tilde{K}_{x} &= \frac{M_{12}}{\Gamma_2}\tilde{K}_{y}\\
\end{align}

\begin{align}
    \left(\begin{array}{c}
        \frac{M_{12}^2+\Gamma_2^2}{\Gamma_2}\tilde{K}_{y}-\frac{\Omega_d}{2}\tilde{K}_{z}\\
        \frac{\Omega_d}{2}\tilde{K}_{y}+\Gamma_1\tilde{K}_{z}
    \end{array}\right)&=
    \left(\begin{array}{c}
        0\\
        R_{se} 
    \end{array}\right)\\
    \left(\begin{array}{c}
        \left(1 + \left(M_{12}/\Gamma_2\right)^2\right)\Gamma_2\tilde{K}_{y}-\frac{\Omega_d}{2}\tilde{K}_{z}\\
        \frac{\Omega_d}{2}\tilde{K}_{y}+\Gamma_1\tilde{K}_{z}
    \end{array}\right)&=
    \left(\begin{array}{c}
        0\\
        R_{se} 
    \end{array}\right)
\end{align}


\begin{align}
   \left(1 + \left(M_{12}/\Gamma_2\right)^2\right)\Gamma_2 \tilde{K}_{y} &=  \frac{\Omega_d}{2}\tilde{K}_{z}\\
    \tilde{K}_{y} &=  \frac{\frac{\Omega_d}{2}}{\left(1 + \left(M_{12}/\Gamma_2\right)^2\right)\Gamma_2}\tilde{K}_{z}\\
\end{align}

\begin{align}
    \frac{\Omega_d}{2}\tilde{K}_{y} + \Gamma_1 \tilde{K}_{z} &= R_{se}\\
    \frac{\Omega_d}{2}\frac{\frac{\Omega_d}{2}}{\left(1 + \left(M_{12}/\Gamma_2\right)^2\right)\Gamma_2}\tilde{K}_{z} + \Gamma_1 \tilde{K}_{z} &= R_{se}\\
    \frac{\frac{\Omega_d^2}{4} + \left(1 + \left(M_{12}/\Gamma_2\right)^2\right)\Gamma_2\Gamma_1}{\left(1 + \left(M_{12}/\Gamma_2\right)^2\right)\Gamma_2} \tilde{K}_{z} &= R_{se}\\
    \frac{\frac{\Omega_d^2}{4\Gamma_2\Gamma_1} + \left(1 + \left(M_{12}/\Gamma_2\right)^2\right)}{\left(1 + \left(M_{12}/\Gamma_2\right)^2\right)/\Gamma_1} \tilde{K}_{z} &= R_{se}\\
     \tilde{K}_{z} &= \frac{R_{se}}{\Gamma_1} \frac{\left(1 + \left(M_{12}/\Gamma_2\right)^2\right)}{\frac{\Omega_d^2}{4\Gamma_2\Gamma_1} + \left(1 + \left(M_{12}/\Gamma_2\right)^2\right)}
\end{align}

\begin{align}
    \tilde{K}_{y} &=  \frac{\Omega_d}{2\Gamma_2}\frac{R_{se}}{\Gamma_1} \frac{1}{\frac{\Omega_d^2}{4\Gamma_2\Gamma_1} + \left(1 + \left(M_{12}/\Gamma_2\right)^2\right)}
\end{align}

\begin{align}
    \tilde{K}_{x} &= \frac{M_{12}}{\Gamma_2}\frac{\Omega_d}{2\Gamma_2}\frac{R_{se}}{\Gamma_1} \frac{1}{\frac{\Omega_d^2}{4\Gamma_2\Gamma_1} + \left(1 + \left(M_{12}/\Gamma_2\right)^2\right)}
\end{align}
    
Finally, the solution to the Bloch equation (eq.(\ref{eq:Bloch_equations})) is

\begin{align}
    \boxed{\left(\begin{array}{c}
        \tilde{K}_{x}\\
        \tilde{K}_{y}\\
        \tilde{K}_{z} 
    \end{array}\right)=
    \frac{\frac{R_{se}}{\Gamma_1}}{\frac{\Omega_d^2}{4\Gamma_2\Gamma_1} + \left(1 + \left(M_{12}/\Gamma_2\right)^2\right)}
    \left(\begin{array}{c}
      \frac{M_{12}}{\Gamma_2}\frac{\Omega_d}{2\Gamma_2} \\
       \frac{\Omega_d}{2\Gamma_2}\\
         \left(1 + \left(M_{12}/\Gamma_2\right)^2\right)
    \end{array}\right)}\label{eq:Bloch_solution}
\end{align}

In the case where $M_{12}=0$ we get

\begin{align}
    \left(\begin{array}{c}
        \tilde{K}_{x}\\
        \tilde{K}_{y}\\
        \tilde{K}_{z} 
    \end{array}\right)=
    \frac{\frac{R_{se}}{\Gamma_1}}{1+\frac{\Omega_d^2}{4\Gamma_2\Gamma_1}}
    \left(\begin{array}{c}
      0 \\
       \frac{\Omega_d}{2\Gamma_2}\\
         1
    \end{array}\right)
\end{align}

Recall that we measure the transverse spin polarization. Let us compute the optimal drive amplitude for maximal $\tilde{K}_y$ polarization

\begin{align}
    \frac{d\tilde{K}_{y}}{d \Omega_d} = \frac{R_{se}}{2\Gamma_2\Gamma_1}\frac{1}{1+\frac{\Omega_d^2}{4\Gamma_2\Gamma_1}} - \frac{R_{se}}{2\Gamma_2\Gamma_1}\frac{\Omega_d^2}{\Gamma_2\Gamma_1} \frac{1}{\left(1+\frac{\Omega_d^2}{4\Gamma_2\Gamma_1}\right)^2}=0
\end{align}

we get the next optimal drive amplitude $$\Omega_d^{\text{optimal}} = \sqrt{\frac{4\Gamma_2\Gamma_1}{3}}$$

for the optimal drive, the $\hat{z}$ polarization would be
$$\tilde{K}_{z} = \frac{3R_{se}}{4\Gamma_1}$$

We can make the same calculation in the case where $M_{12}\neq 0$

\begin{align}
    \frac{d\tilde{K}_{y}}{d \Omega_d} = \frac{R_{se}}{2\Gamma_2\Gamma_1}\frac{1}{\frac{\Omega_d^2}{4\Gamma_2\Gamma_1} + \left(1 + \left(M_{12}/\Gamma_2\right)^2\right)} - \frac{R_{se}}{2\Gamma_2\Gamma_1}\frac{\Omega_d^2}{\Gamma_2\Gamma_1} \frac{1}{\left(\frac{\Omega_d^2}{4\Gamma_2\Gamma_1} + \left(1 + \left(M_{12}/\Gamma_2\right)^2\right)\right)^2}=0
\end{align}

in that case the optimal drive amplitude is

$$\Omega_d^{\text{optimal}} =\sqrt{\frac{4\Gamma_2\Gamma_1}{3}}\sqrt{1 + \left(M_{12}/\Gamma_2\right)^2}$$

Just for sports we can also solve the case of drive in $\hat{y}$, the Bloch equations in that case are

\begin{align}
    \boxed{\frac{d}{dt}\left(\begin{array}{c}
        \tilde{K}_{x}\\
        \tilde{K}_{y}\\
        \tilde{K}_{z}
    \end{array}\right)=
   \left(\begin{matrix}
    -\Gamma_2 & M_{12}    &  -\frac{\Omega_d}{2} \\
    M_{21} &  - \Gamma_2 &  0 \\ \frac{\Omega_d}{2} &   0 &  -\Gamma_1 
    \end{matrix}\right)
    \left(\begin{array}{c}
        \tilde{K}_{x}\\
        \tilde{K}_{y}\\
        \tilde{K}_{z}
    \end{array}\right) + 
    \left(\begin{matrix}
         0  \\
         0  \\
         R_{se} 
    \end{matrix}\right)}\label{eq:Bloch_equations_y_drive}
\end{align}

then the steady state solution would be

\begin{align}
    \left(\begin{array}{c}
        \Gamma_2\tilde{K}_{x} - M_{12}\tilde{K}_{y} + \frac{\Omega_d}{2}\tilde{K}_{z}\\
        M_{12}\tilde{K}_{x}+\Gamma_2\tilde{K}_{y}\\
        -\frac{\Omega_d}{2}\tilde{K}_{x}+\Gamma_1\tilde{K}_{z}
    \end{array}\right)&=
    \left(\begin{array}{c}
        0\\
        0\\
        R_{se} 
    \end{array}\right)
\end{align}


$$\tilde{K}_y = - \frac{M_{12}}{\Gamma_2}\tilde{K}_x$$

\begin{align}
     \left(\begin{array}{c}
        \frac{\Gamma_2^2 + M_{12}^2}{\Gamma_2}\tilde{K}_{x} + \frac{\Omega_d}{2}\tilde{K}_{z}\\
        -\frac{\Omega_d}{2}\tilde{K}_{x}+\Gamma_1\tilde{K}_{z}
    \end{array}\right)&=
    \left(\begin{array}{c}
        0\\
        R_{se} 
    \end{array}\right)\\
    \left(\begin{array}{c}
        \left(1 + \left(M_{12}/\Gamma_2\right)^2\right)\Gamma_2\tilde{K}_{x} + \frac{\Omega_d}{2}\tilde{K}_{z}\\
        -\frac{\Omega_d}{2}\tilde{K}_{x}+\Gamma_1\tilde{K}_{z}
    \end{array}\right)&=
    \left(\begin{array}{c}
        0\\
        R_{se} 
    \end{array}\right)
\end{align}

$$\tilde{K}_{x} = -\frac{\Omega_d}{2}\frac{1}{\left(1 + \left(M_{12}/\Gamma_2\right)^2\right)\Gamma_2}\tilde{K}_{z}$$

$$\tilde{K}_{z} = \frac{R_{se}}{\Gamma_1} \frac{\left(1 + \left(M_{12}/\Gamma_2\right)^2\right)}{\frac{\Omega_d^2}{4\Gamma_2\Gamma_1} + \left(1 + \left(M_{12}/\Gamma_2\right)^2\right)} $$

so finally,

\begin{align}
    \boxed{\left(\begin{array}{c}
        \tilde{K}_{x}\\
        \tilde{K}_{y}\\
        \tilde{K}_{z} 
    \end{array}\right)=
    \frac{\frac{R_{se}}{\Gamma_1}}{\frac{\Omega_d^2}{4\Gamma_2\Gamma_1} + \left(1 + \left(M_{12}/\Gamma_2\right)^2\right)}
    \left(\begin{array}{c}
    -\frac{\Omega_d}{2\Gamma_2}\\
      \frac{M_{12}}{\Gamma_2}\frac{\Omega_d}{2\Gamma_2} \\
         \left(1 + \left(M_{12}/\Gamma_2\right)^2\right)
    \end{array}\right)}\label{eq:Bloch_solution_y_drive}
\end{align}



\section{NMR Gyroscope (NMRG)}
Using the Xenon NMR model (the Bloch equations) we can implement a gyroscope which measures rotation along the primary axis of the system (in the case above: $\hat{z}$). 
Following Walker's notation in \cite{walker2016spin}, we break the spin dynamics into two components, parallel and perpendicular to the primary axis $\hat{z}$  $$\mathbf{K}=K_z\hat{z}+\mathbf{K}_{\perp}$$

describing $\mathbf{K}_{\perp}$ using a phasor notation
$$K_{+} = K_x + iK_y = K_{\perp}e^{-i\phi}$$
where $$K_{\perp}=\sqrt{K_x^2+K_y^2}\quad, \quad \phi=\arctan{\left(\frac{K_y}{K_x}\right)}$$

% Solving for the perpendicular polarization in the rotating frame we get

% \begin{align}
%     \frac{d\tilde{K}_{+}}{dt} &= \frac{d\tilde{K}_{x}}{dt}+i\frac{d\tilde{K}_{y}}{dt}\\
%     &=-\Gamma_2 \tilde{K}_x + M_{12} \tilde{K}_y - \frac{\Omega_d}{2} \tilde{K}_z \\&- iM_{12} \tilde{K}_{x} - i\Gamma_2\tilde{K}_{y}\\
% \end{align}


Now, if we solve for $\phi$ in the rotating frame of the drive, from eq.(\ref{eq:Bloch_solution_y_drive}) we get

\begin{align}
    \phi = \arctan{\left(\frac{\tilde{K}_y}{\tilde{K}_x}\right)}=\arctan{\left(-\frac{M_{12}}{\Gamma_2}\right)}
\end{align}

for small phase, where $M_{12}\ll 1$ the phase can be approximated with it zeroth order Taylor expansion $$\phi= -\frac{M_{12}}{\Gamma_2}$$

unfolding $M_{12}$ we get
$$\phi= -\frac{\gamma_{Xe}B_0}{\Gamma_2}-\frac{\gamma_{Xe}b_{KS}S_z}{\Gamma_2}-\frac{\omega_r}{\Gamma_2}+\frac{\omega_{rf}}{\Gamma_2}$$

In a real life NMR system we would also have some magnetic noise in the primary axis $\gamma_{Xe} B_{noise}$, so the phase would be
$$\phi= -\frac{\gamma_{Xe}B_0}{\Gamma_2}-\frac{\gamma_{Xe}B_{noise}}{\Gamma_2}-\frac{\gamma_{Xe}b_{KS}S_z}{\Gamma_2}-\frac{\omega_r}{\Gamma_2}+\frac{\omega_{rf}}{\Gamma_2}$$

all these parameters could in general be time dependent. Notice that in the phase equation we have two unknown parameters: fluctuations of the magnetic field $B_{noise}$ and the world rotation $\omega_r$. In order two extract the world rotation from the phase measurements we need to introduce another specie.  


\subsection{Dual species open-loop NMRG scheme}
Having system with two species, Xe129 and Xe131, we can compute the world rotation as follows

\begin{align}
    \phi_{129} &= -\frac{\gamma^{129}_{Xe}B_0}{\Gamma_2^{129}}-\frac{\gamma^{129}_{Xe}B_{noise}}{\Gamma_2^{129}}-\frac{\gamma^{129}_{Xe}b_{KS}S_z}{\Gamma_2^{129}}-\frac{\omega_r}{\Gamma_2^{129}}+\frac{\omega^{129}_{rf}}{\Gamma_2^{129}}\\
    \phi_{131} &= -\frac{\gamma^{131}_{Xe}B_0}{\Gamma_2^{131}}-\frac{\gamma^{131}_{Xe}B_{noise}}{\Gamma_2^{131}}-\frac{\gamma^{131}_{Xe}b_{KS}S_z}{\Gamma_2^{131}}-\frac{\omega_r}{\Gamma_2^{131}}+\frac{\omega^{131}_{rf}}{\Gamma_2^{131}}
\end{align}

Then, subtracting the two equations we get

\begin{align}
    \frac{1}{\gamma^{129}_{Xe}}\left(\phi_{129}\Gamma_2^{129}-\omega^{129}_{rf}\right) - \frac{1}{\gamma^{131}_{Xe}}\left(\phi_{131}\Gamma_2^{131}-\omega^{131}_{rf}\right) &= \omega_r \left(\frac{1}{\gamma^{131}_{Xe}} - \frac{1}{\gamma^{129}_{Xe}}\right)
\end{align}

and finally the world rotation in an open-loop setting is given by 

\begin{align}
    \omega_r &= \left[\frac{ \gamma^{129}_{Xe} \gamma^{131}_{Xe} }{\gamma^{129}_{Xe} - \gamma^{131}_{Xe}}\right]\left[\frac{1}{\gamma^{129}_{Xe}}\left(\phi_{129}\Gamma_2^{129}-\omega^{129}_{rf}\right) - \frac{1}{\gamma^{131}_{Xe}}\left(\phi_{131}\Gamma_2^{131}-\omega^{131}_{rf}\right)\right]
\end{align}


\subsection{Dual species closed-loop NMRG scheme}


\bibliographystyle{alpha}
\bibliography{my_bib}

\end{document}
